\lecture{6: Monty Hall, Simpson's Paradox}
\textbf{Key Topics:} Monty Hall, Simpson's Paradox

\subsection*{Lecture Summary}
\begin{itemize}
    \item Monty Hall
    \item Simpson's Paradox
\end{itemize}

\subsection*{Monty Hall}
On the game show Let’s Make a Deal, hosted by Monty Hall, a contestant chooses one of three closed doors, two of which have a goat behind them and one of which has a car. Monty, who knows where the car is, then opens one of the two remaining doors. The door he opens always has a goat behind it (he never reveals the car!). If he has a choice, then he picks a door at random with equal probabilities. Monty then offers the contestant the option of switching to the other unopened door. If the contestant’s goal is to get the car, should she switch doors?
\begin{figure}[ht]
    \centering
    \includegraphics[width=0.5\textwidth]{lectures/images/lect_06_doors.png}
\end{figure}\\
\noindent
\textit{Solution:}

Let’s label the doors 1 through 3. Without loss of generality, we can assume the contestant picked door 1 (if she didn’t pick door 1, we could simply relabel the doors, or rewrite this solution with the door numbers permuted). Monty opens a door, revealing a goat. As the contestant decides whether or not to switch to the remaining unopened door, what does she really wish she knew? Naturally, her decision would be a lot easier if she knew where the car was! This suggests that we should condition on the location of the car. Let $C_i$ be the event that the car is behind door $i$, for $i = 1, 2, 3$. By the law of total probability,
$$
P(\text{get car}) = P(\text{get car}|C_1)\cdot \frac{1}{3} + P(\text{get car}|C_2)\cdot \frac{1}{3} + P(\text{get car}|C_3)\cdot \frac{1}{3}.
$$

Suppose the contestant employs the switching strategy. If the car is behind door 1, then switching will fail, so $P(\text{get car}|C_1) = 0$. If the car is behind door 2 or 3, then because Monty always reveals a goat, the remaining unopened door must contain the car, so switching will succeed. Thus,
$$
P(\text{get car}) = 0 \cdot \frac{1}{3} + 1\cdot \frac{1}{3} + 1\cdot \frac{1}{3} = \frac{2}{3},
$$
so the switching strategy succeeds 2/3 of the time. The contestant should switch to the other door.

There’s a subtlety though, which is that when the contestant chooses whether to switch, she also knows which door Monty opened. We showed that the unconditional probability of success is 2/3 (when following the switching strategy), but let’s also show that the conditional probability of success for switching, given the information that Monty provides, is also 2/3.

Let $M_j$ be the event that Monty opens door $j$, for $j = 2, 3$. Then
$$
P(\text{get car}) = P(\text{get car}|M_2)P(M_2) + P(\text{get car}|M_3)P(M_3),
$$
where by symmetry $P(M_2) = P(M_3) = 1/2$ and $P(\text{get car}|M_2) = P(\text{get car}|M_3)$. The symmetry here is that there is nothing in the statement of the problem that distinguishes between door 2 and door 3; in contrast, to the Lazy Monty Hall which considers a scenario where Monty enjoys opening door 2 more than he enjoys opening door 3.

Let $x = P(\text{get car}|M_2) = P(\text{get car}|M_3)$. Plugging in what we know,
$$
\frac{2}{3} = P(\text{get car}) = \frac{x}{2} + \frac{x}{2} = x,
$$
as claimed.

Bayes’ rule also works nicely for finding the conditional probability of success using the switching strategy, given the evidence. Suppose that Monty opens door 2. Using the notation and results above,
$$
P(C_1|M_2) = \frac{P(M_2|C_1)P(C_1)}{P(M_2)} = \frac{(1/2)(1/3)}{1/2} = \frac{1}{3}.
$$

So given that Monty opens door 2, there is a 1/3 chance that the contestant’s original choice of door has the car, which means that there is a 2/3 chance that the switching strategy will succeed.

Many people, upon seeing this problem for the first time, argue that there is no advantage to switching: “There are two doors remaining, and one of them has the car, so the chances are 50-50.” After the last chapter, we recognize that this argument misapplies the naive definition of probability. Yet the naive definition, even when inappropriate, has a powerful hold on people’s intuitions.

\subsection*{Simpson's Paradox}
Two doctors, Dr. Hibbert and Dr. Nick, each perform two types of surgeries: heart surgery and Band-Aid removal. Each surgery can be either a success or a failure. The two doctors’ respective records are given in the following tables.\\

\begin{minipage}{0.45\linewidth}
\centering
\begin{tabular}{lcc}
\toprule
 & Heart & Band-Aid \\
\midrule
Success & 70 & 10 \\
Failure & 20 & 0 \\
\bottomrule
\end{tabular}

\vspace{2mm}
Dr. Hibbert
\end{minipage}
\hfill
\begin{minipage}{0.45\linewidth}
\centering
\begin{tabular}{lcc}
\toprule
 & Heart & Band-Aid \\
\midrule
Success & 2 & 81 \\
Failure & 8 & 9 \\
\bottomrule
\end{tabular}

\vspace{2mm}
Dr. Nick
\end{minipage}\\

Dr. Hibbert had a higher success rate than Dr. Nick in heart surgeries: 70 out of 90 versus 2 out of 10. Dr. Hibbert also had a higher success rate in Band-Aid removal: 10 out of 10 versus 81 out of 90. But if we aggregate across the two types of surgeries to compare overall surgery success rates, Dr. Hibbert was successful in 80 out of 100 surgeries while Dr. Nick was successful in 83 out of 100 surgeries: Dr. Nick’s overall success rate is higher!

What’s happening is that Dr. Hibbert, presumably due to his reputation as the superior doctor, is performing a greater number of heart surgeries, which are inherently riskier than Band-Aid removals. His overall success rate is lower not because of lesser skill on any particular type of surgery, but because a larger fraction of his surgeries are risky.

Let’s use event notation to make this precise. For events $A$, $B$, and $C$, we say that we have a \textit{Simpson’s paradox} if
\begin{align*}
    P(A|B, C) &< P(A|B^c, C)\\
    P(A|B, C^c) &< P(A|B^c, C^c),
\end{align*}
but
$$
P(A|B) > P(A|B^c).
$$

In this case, let $A$ be the event of a successful surgery, $B$ be the event that Dr. Nick is the surgeon, and $C$ be the event that the surgery is a heart surgery. The conditions for Simpson’s paradox are fulfilled because the probability of a successful surgery is lower under Dr. Nick than under Dr. Hibbert whether we condition on heart surgery or on Band-Aid removal, but the overall probability of success is higher for Dr. Nick.

The law of total probability tells us mathematically why this can happen:
\begin{align*}
    P(A|B) &= P(A|C, B)P(C|B) + P(A|C^c, B)P(C^c|B)\\
    P(A|B^c) &= P(A|C, B^c)P(C|B^c) + P(A|C^c, B^c)P(C^c|B^c).
\end{align*}

The above equations express $P(A|B)$ as a weighted average of $P(A|C, B)$ and $P(A|C^c, B)$, and $P(A|B^c)$ as a weighted average of $P(A|C, B^c)$ and $P(A|C^c, B^c)$. If the corresponding weights were the same in both of these weighted averages, then Simpson’s paradox could not occur. But the weights here are \textit{different}:
$$
P(C|B) < P(C|B^c) \text{ and } P(C^c|B) > P(C^c|B^c),
$$
since Dr. Nick is much less likely than Dr. Hibbert to be performing a heart surgery.

Although we have
$$
P(A|C, B) < P(A|C, B^c)
$$
and
$$
P(A|C^c, B) < P(A|C^c, B^c),
$$
the fact that the weights are so different results in the inequality flipping when we do not condition on whether or not $C$ occurred:
$$
P(A|B) > P(A|B^c).
$$

Numerically, the two weighted averages are
\begin{align*}
    P(A|B) &= 0.83 = (2/10) \cdot 0.1 + (81/90) \cdot 0.9\\
    P(A|B^c) &= 0.80 = (70/90) \cdot 0.9 + (10/10) \cdot 0.1.
\end{align*}

The first equation (corresponding to Dr. Nick) puts much more weight on the second term (corresponding to the easier surgery) than does the second equation.

Aggregation across different types of surgeries presents a misleading picture of the doctors’ abilities because we lose the information about which doctor tends to perform which type of surgery. When we think \textit{confounding variables} like surgery type could be at play, we should examine the disaggregated data to see what is really going on.